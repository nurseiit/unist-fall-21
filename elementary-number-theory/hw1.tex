\documentclass[12pt]{article}
\usepackage{amsmath}
\usepackage{amssymb}
\usepackage{mathabx}
\usepackage{graphicx}
\usepackage{hyperref}
\usepackage[latin1]{inputenc}
\usepackage{mathtools}
\DeclarePairedDelimiter{\ceil}{\lceil}{\rceil}
\DeclarePairedDelimiter{\floor}{\lfloor}{\rfloor}

\title{MTH26001 Assignment 1} 
\author{Nurseiit Abdimomyn -- 20172001}
\date{16/09/2021}

\begin{document}
\maketitle

\begin{enumerate}
  \item
    (Sec 2.2 - 8) Observe that any term of the form $111...111$ can be written as
    $100m + 11 = 4 * 25m + 11$ and $4 * 25m + 11 \equiv 3 \pmod 4$ for some $m$. (a)
    
    As all terms of the form $111...111$ are odd, so must be the square roots of them (i.e $n^2$ is odd, then so is $n$).
    Transform $n = 2*k + 1$ and $n^2 = 4*k^2 + 4k + 1$ which by modulo $4$ is not $3$ which we got from (a).
    To be more precise, $n^2 = 4*k^2 + 4k + 1 \equiv 1 \neq 3 \pmod 4$.

  \item
    (Sec 2.2 - 9) For a number to be both a square and a cube it must of the form $n^6$.
    We could prove that $n^6$ can only be either of $7k$ or $7k + 1$ by Fermat's little theorem
    picking $p = 7$. But let's keep it short and observe that the congruent clasess of $n^6 \pmod 7$
    is a set of $\{ 0, 1 \}$.

  \item
  (Sec 2.3 - 4d) Let's first establish a base case for induction:
    
    $21 \divides 4^{n + 1} + 5^{2n - 1} = 4^{1 + 1} + 5^{2 - 1} = 21$ for $n = 1$.

  Assume it holds true for some $n$. Then it should follow for $n + 1$ that
  $21 \divides 4^{(n + 1) + 1} + 5^{2*(n + 1) - 1} = 4^{n + 2} + 5^{2n + 1} = 
  4 * 4^{n + 1} + 5^2 * 5^{2n - 1} = 4 * (4^{n + 1} + 5^{2n - 1}) + 21 * 5^{2n - 1}$.
    
  \item
  (Sec 2.3 - 8b) Prove that $k * (k + 1) * (k + 2) * (k + 3) = n^2 - 1$ for some $k, n$;

  Let's rearrange it like so: $k * (k + 3) * (k + 1) * (k + 2) = (k^2 + 3k) * (k^2 + 3k + 2) = 
  ((k^2 + 3k + 1) - 1) * ((k^2 + 3k + 1) + 1) = (k^2 + 3k + 1)^2 - 1$ so $n$ is $(k^2 + 3k + 1)$.

  \item
  (Sec 2.3 - 15) We know that $gcd(a, b) = gcd(a, b - a)$.
  
  So, $gcd(2a - 3b, 4a - 5b) =
  gcd(2a - 3b, 4a - 5b - (2a - 3b)) = gcd(2a - 3b, 4a - 5b - (2a - 3b) - (2a - 3b)) =
  gcd(2a - 3b, 4a - 5b - 2 * (2a - 3b)) = gcd(2a - 3b, b)$.
  
  Thus, $gcd(2a - 3b, 4a - 5b) \divides b$. Let's assign $b = -1$, then it follows that $d = 1$ and
  $gcd(2a - 3b, 4a - 5b) = gcd(2a + 3, 4a + 5) = 1$.

  \item
  (Sec 2.4 - 4c) Let $d = gcd(a + b, a^2 + b^2)$. By definition of $gcd$, we know that both $d \divides (a + b)$
  and $d \divides (a^2 + b^2)$. Then, by rearranging $a^2 + b^2$ as $(a + b) * (a - b) + 2b^2$ and $(a + b) * (b - a) + 2a^2$ we learn that
  $d \divides 2b^2$ and $d \divides 2a^2$.
  
  Consequently, $d \divides 2a^2 \land d \divides 2b^2 \implies d \divides gcd(2a^2, 2b^2)$.
  $gcd(2a^2, 2b^2) = 2 * gcd(a^2, b^2) = 2 * (gcd(a, b))^2 = 2 * 1 ^ 2 = 2$.

  Thus, $d$ is a divisor of $2$ (i.e $1$ and $2$).

  ps. proof of $gcd(a^2, b^2) = (gcd(a, b))^2$ is omitted for clarity.

  \item
  (Sec 2.4 - 6) If $gcd(a, b) = 1$, prove that $gcd(a + b, ab) = 1$.

  Let $d = gcd(a + b, ab)$. Now, since $d \divides a + b$ and $d \divides ab$,
  it follows that $d \divides a * (a + b) - ab = a^2$ -- similar for $d \divides b^2$.

  Similar to the proof in (Sec 2.4 - 4c), $d \divides a^2 \land d \divides b^2 \implies
  d \divides gcd(a^2, b^2) = (gcd(a, b))^2 = 1^2 = 1$. So $d = gcd(a + b, ab) = 1$.

  \item
  (Sec 2.5 - 3b) $54x + 21y = 906$. $gcd(54, 21) = 3 \divides 906$, so there's a solution.
  Simplify to $18x + 7y = 302$. By brute force find that $x = 2, y = 38$ is one solution.

  Substituting $7$ for $x$ for $16$ in $y$ we find other solutions. So $x = 2, y = 38; x = 16, y = 2; x = 9, y = 20;$

  \item
  (Sec 2.5 - 5c) $6x + 9y = 126$ and $9x + 6y = 114$, then subtracting one from the other we get
  $3y - 3x = 12$, dividing each side by $3$ is $y - x = 4$, $y = x + 4$.

  Substituting for $y$ in the initial equation we get $9x + 6*(x + 4) = 114$. Solving for $x = 6$.
  Then $y = x + 4$, so $y = 6 + 4 = 10$. Thus, $x = 6, y = 10$.
\end{enumerate}

\end{document}
