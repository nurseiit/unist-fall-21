\documentclass[12pt]{article}
\usepackage{amsmath}
\usepackage{amssymb}
\usepackage{mathabx}
\usepackage{graphicx}
\usepackage{hyperref}
\usepackage[latin1]{inputenc}
\usepackage{mathtools}
\DeclarePairedDelimiter{\ceil}{\lceil}{\rceil}
\DeclarePairedDelimiter{\floor}{\lfloor}{\rfloor}

\title{MTH26001 Assignment 3} 
\author{Nurseiit Abdimomyn -- 20172001}
\date{13/10/2021}

\begin{document}
\maketitle

\begin{enumerate}
  \item
    (Sec 4.2 - 3)
    
    Given $a \equiv b \pmod n$ implies that $k*n = a - b$ for some $k$.
    Because $gcd(b, n) \divides b$ and $gcd(b, n) \divides n$, it's also
    true that $gcd(b, n) \divides a = k*n + b$.

    Since $gcd(b, n) \divides a$, it should imply that $gcd(b, n) \divides gcd(a, n)$.

    Then, we could also prove similar for $b = a - k*n$. Because $gcd(a, n) \divides a$ and $gcd(a, n) \divides n$, it's should
    follow that $gcd(a, n) \divides b = a - k*n$.

    Since $gcd(a, n) \divides b$, it should imply that $gcd(a, n) \divides gcd(b, n)$.

    Thus, $gcd(a, n) \divides gcd(b, n)$ and $gcd(b, n) \divides gcd(a, n)$ implies that $gcd(a, n) = gcd(b, n)$.

  \item
    (Sec 4.2 - 5)

    $53^{103} \equiv 1^{103} \pmod 13$ and $103^{53} \equiv (-1)^{53} \pmod {13}$.

    Thus, $53^{103} + 103^{53} \equiv 1^{103} + (-1)^{53} \equiv 0 \pmod {13}$.

    Also, $53^{103} \equiv (-1)^{103} \pmod 3$ and $103^{53} \equiv 1^{53} \pmod 3$.

    Thus, $53^{103} + 103^{53} \equiv (-1)^{103} + 1^{53} \equiv 0 \pmod 3$. \\ \\

    $111^{333} + 4^{111} \equiv (-1)^{333} + 4^{111} \pmod 7$

    $\equiv (-1)^{333} + (4^3)^{37} \pmod 7$

    $\equiv (-1)^{333} + 1^{37} \equiv 0 \pmod 7$ [as $64 \equiv 1 \pmod 7$].

  \item
    (Sec 4.2 - 11)

    ${0, 2^0, 2^1, ..., 2^9} \equiv {0, 1, 2, 4, 8, 16, 32, 64, 128, 256, 512}$
    
    $\equiv {0, 1, 2, 4, 8, 5, 10, 9, 7, 3, 6} \pmod{11}$, thus a complete residue set modulo $11$. \\ \\

    But, $0, 1^2, 2^2, ..., 10^2$ is not a complete residue set modulo $11$ because $1^2 \equiv 10^2 \pmod {11}$.

  \item
    (Sec 4.3 - 3)

    Let's look for the period of $9^n \pmod {100}$:

    so powers of nine modulo hundred are $1, 9, 81, 29, 61, 49, 41, 69, 21, 89, 1$, with a period of $10$.

    Since $9^9 \equiv 9 \pmod {10}$ (here $10$ is our period), $9^{9^9} \equiv 89 \pmod {100}$.

  \item
    (Sec 4.3 - 10)

    All numbers $n$ with digits adding up to $15$ are $n \equiv 6 \pmod 9$. However,
    for all $a$, it's true that $a^3 \equiv 0, 1, 8 \pmod 9$ and $a^2 \equiv 1, 4, 7 \pmod 9$.

    Thus, $n$ can not be a cube or a square.

  \item
    (Sec 4.3 - 11)

    $495 = 5 * 9 * 11$, the resulting number should be divisible by all three $5$, $9$, and $11$.

    Since $273x49y5$ ends with $5$, it already is divisible by $5$.

    In case of $9$, sum of digits should be divisible by $9$:

    $2 + 7 + 3 + x + 4 + 9 + y + 5 \equiv 0 \pmod 9$

    $\equiv 3 + x + y \equiv 0 \pmod 9$.

    In case of $11$, alternating sum of digits should be divisible by $11$:

    $2 + 3 + 4 + y - (7 + x + 9 + 5) \equiv 0 \pmod {11}$

    $x - y + 1 \equiv 0 \pmod{11}$.

    Thus, since $0 \leq x, y \leq 9$, there is only one solution: 

    $x = 7, y = 8$.

    Indeed, $27374985 \equiv 0 \pmod{495}$.

  \item
    (Sec 4.3 - 25)

    Let's first prove that for any prime $p$, that it's either of form $6k + 1$ or $6k - 1$.

    Considering all numbers in a form $6k + r$ we see that:

    $6k + 0$ is divisible by $6$.

    $6k + 1$ has no apparent divisor.

    $6k + 2$ is divisible by $2$.

    $6k + 3$ is divisible by $3$.

    $6k + 4$ is divisible by $2$.

    $6k + 5$ has no apparent divisor [i.e $6k - 1 \pmod 6$]. \\

    Let's prove the original problem for when $p = 6k + 1$ first by induction,

    so it becomes: $10^{2(6k + 1)} - 10^{6k + 1} + 1 \equiv 0 \pmod{13}$.

    (a). $k = 1$. $10^{14} - 10^7 + 1 \equiv 9 - 10 + 1 \equiv 0 \pmod{13}$.

    (b). Assume $10^{2(6k + 1)} - 10^{6k + 1} + 1 \equiv 0 \pmod{13}$,

    then, $10^{2(6(k + 1) + 1)} - 10^{6(k + 1) + 1} + 1 \equiv$

    $\equiv 10^{2(6k + 7)} - 10^{6k + 7} + 1 \pmod{13}$

    $\equiv 10^{2*6} * 10^{2(6k + 1)} - 10^6 * 10^{6k + 1} + 1 \equiv 0 \pmod{13}$. \\

    Now, similarly for $p = 6k - 1$,

    $10^{2(6k - 1)} - 10^{6k - 1} + 1 \equiv 0 \pmod{13}$.

    (a). $k = 1$. $10^{10} - 10^5 + 1 \equiv 3 - 4 + 1 \equiv 0 \pmod{13}$.

    (b). Assume $10^{2(6k - 1)} - 10^{6k - 1} + 1 \equiv 0 \pmod{13}$,

    then, $10^{2(6(k + 1) - 1)} - 10^{6(k + 1) - 1} + 1 \equiv$

    $\equiv 10^{2(6k + 5)} - 10^{6k + 5} + 1 \pmod{13}$

    $\equiv 10^{2*6} * 10^{2(6k - 1)} - 10^6 * 10^{6k - 1} + 1 \equiv 0 \pmod{13}$.

\end{enumerate}

\end{document}
